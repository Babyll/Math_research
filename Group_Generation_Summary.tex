\documentclass[12pt]{article}

\setlength{\oddsidemargin}{0.5in}
\setlength{\evensidemargin}{0.5in}
\setlength{\topmargin}{0in}
\setlength{\headheight}{0in}
\setlength{\headsep}{0in}
\setlength{\topskip}{0in}
\setlength{\textheight}{8.5in}
\setlength{\textwidth}{5.3in}
\setlength{\footskip}{1in}

\usepackage{latexsym}
\usepackage{amsfonts,amssymb,amsmath}
\usepackage{graphics, color}

\newcommand{\C}{\mathbb{C}}
\newcommand{\Q}{\mathbb{Q}}
\newcommand{\R}{\mathbb{R}}
\newcommand{\Z}{\mathbb{Z}}

\newtheorem{definition}{Definition}[section]

\newtheorem{lemma}{Lemma}[section]

\newtheorem{example}{Example}[section]

\newtheorem{proposition}{Proposition}[section]

\newtheorem{theorem}{Theorem}[section]

\newtheorem{claim}{Claim}[section]

\newtheorem{corollary}{Corollary}[section]
\begin{document}
\title{Summary of Findings from Studying Group Generated by A and B}
\author{Saad Khalid}
\date{December 1\textsuperscript{st}, 2015}

\maketitle

\section*{Summary}

The first step was proving that: \\

\[
	A^j =
	\begin{bmatrix}
		\cos\frac{2\pi}{n} & -\sin\frac{2\pi}{n} \\
		\\
		\sin\frac{2\pi}{n} &  \cos\frac{2\pi}{n}
	\end{bmatrix}^j
	=
	\begin{bmatrix}
		\cos\frac{2j\pi}{n} & -\sin\frac{2j\pi}{n} \\
		\\
		\sin\frac{2j\pi}{n} &  \cos\frac{2j\pi}{n}
	\end{bmatrix}
\]

After this, it is simple enough to also prove that $A^j = A^{j+n}$. This means that $A^j$ repeats for every $n$ increase in the exponent. \\

It is fairly obvious that the group generated by $B^k$ has 2 elements

We then prove that $A^jB = BA^{-j}$ \\

Taking all these factors together, we can see that $A^j$ has $n$ unique values, as j = 1,2,3,4,...,n. Even though the group generated by A and B must consider both $A^jB^k$ as well as $B^kA^j$, we know that the latter can be written instead as $A^{-j}B^k$, and then we can we can make the exponent of $A$ positive by adding the appropriate multiple of $n$.




\end{document}





