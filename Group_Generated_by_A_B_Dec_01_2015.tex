\documentclass[12pt]{article}

\setlength{\oddsidemargin}{0.5in}
\setlength{\evensidemargin}{0.5in}
\setlength{\topmargin}{0in}
\setlength{\headheight}{0in}
\setlength{\headsep}{0in}
\setlength{\topskip}{0in}
\setlength{\textheight}{8.5in}
\setlength{\textwidth}{5.3in}
\setlength{\footskip}{1in}

\usepackage{latexsym}
\usepackage{amsfonts,amssymb,amsmath}
\usepackage{graphics, color}

\newcommand{\C}{\mathbb{C}}
\newcommand{\Q}{\mathbb{Q}}
\newcommand{\R}{\mathbb{R}}
\newcommand{\Z}{\mathbb{Z}}

\newtheorem{definition}{Definition}[section]

\newtheorem{lemma}{Lemma}[section]

\newtheorem{example}{Example}[section]

\newtheorem{proposition}{Proposition}[section]

\newtheorem{theorem}{Theorem}[section]

\newtheorem{claim}{Claim}[section]

\newtheorem{corollary}{Corollary}[section]
\begin{document}

\title{Number of Elements in the Group Generated by A and B}
\author{Saad Khalid}
\date{December 1\textsuperscript{st}, 2015}

\maketitle

\section*{Proving the group generated by $A^j$ has n elements}

\[
	A = 
	\begin{bmatrix}
		\cos\frac{2\pi}{n} & -\sin\frac{2\pi}{n} \\
		\\
		\sin\frac{2\pi}{n} &  \cos\frac{2\pi}{n}
	\end{bmatrix}
\]
\subsection*{Lemma}

Before our main proof, we will propose that: \\

\[
	A^j = 
	\begin{bmatrix}
		\cos\frac{2j\pi}{n} & -\sin\frac{2j\pi}{n} \\
		\\
		\sin\frac{2j\pi}{n} &  \cos\frac{2j\pi}{n}
	\end{bmatrix}
\]

for $j\in\mathbb{Z}^{+}$
\\

\subsubsection*{Proof}

We begin by noticing that: \\
\[
	A^2 = 
	\begin{bmatrix}
		\cos\frac{2\pi}{n} & -\sin\frac{2\pi}{n} \\
		\\
		\sin\frac{2\pi}{n} &  \cos\frac{2\pi}{n}
	\end{bmatrix}
	\begin{bmatrix}
		\cos\frac{2\pi}{n} & -\sin\frac{2\pi}{n} \\
		\\
		\sin\frac{2\pi}{n} &  \cos\frac{2\pi}{n}
	\end{bmatrix}
\]
\[
	=
	\begin{bmatrix}
		\cos^2(\frac{2\pi}{n}) - \sin^2(\frac{2\pi}{n}) & -2\cos (\frac{2\pi}{n})\sin (\frac{2\pi}{n}) \\
		\\
		-2\cos (\frac{2\pi}{n}) \sin (\frac{2\pi}{n}) & \cos^2(\frac{2\pi}{n}) - \sin^2(\frac{2\pi}{n})  
	\end{bmatrix}
\]
Using the Double Angle Formulas, we know:

\[
	\cos 2u = \cos^2u - \sin^2u
\]
\[
	\sin 2u = 2\sin u\cos u
\]

So, 
\[
	\begin{bmatrix}
		\cos^2(\frac{2\pi}{n}) - \sin^2(\frac{2\pi}{n}) & -2\cos (\frac{2\pi}{n})\sin (\frac{2\pi}{n}) \\
		\\
		2\cos (\frac{2\pi}{n}) \sin (\frac{2\pi}{n}) & \cos^2(\frac{2\pi}{n}) - \sin^2(\frac{2\pi}{n})  
	\end{bmatrix}
	=
	\begin{bmatrix}
		\cos\frac{4\pi}{n} & -\sin\frac{4\pi}{n} \\
		\\
		\sin\frac{4\pi}{n} &  \cos\frac{4\pi}{n}
	\end{bmatrix}
\]

Now, let's assume this formula is true for $j = 1,2,3,...,x-1$ \\
Then, \\
\[
	A^j = AA^{j-1}
	=
	\begin{bmatrix}
		\cos\frac{2\pi}{n} & -\sin\frac{2\pi}{n} \\
		\\
		\sin\frac{2\pi}{n} &  \cos\frac{2\pi}{n}
	\end{bmatrix}
	\begin{bmatrix}
		\cos\frac{2(j-1)\pi}{n} & -\sin\frac{2(j-1)\pi}{n} \\
		\\
		\sin\frac{2(j-1)\pi}{n} &  \cos\frac{2(j-1)\pi}{n}
	\end{bmatrix}
\]
Using the Sum-Difference Formulas for Trigonometric Functions:
\[
	=
	\begin{bmatrix}
		\cos (\frac{2\pi + 2\pi j - 2\pi}{n} & -\sin (\frac{2\pi + 2j\pi - 2\pi}{n} \\
		\\
		\sin (\frac{2\pi + 2j\pi - 2\pi}{n} & \cos (\frac{2\pi + 2\pi j - 2\pi}{n}  
	\end{bmatrix}
	\]
	Now using the Double Angle Formulas:

	\[
		=
		\begin{bmatrix}
			\cos\frac{2j\pi}{n} & -\sin\frac{2j\pi}{n} \\
			\\
			\sin\frac{2j\pi}{n} &  \cos\frac{2j\pi}{n}
		\end{bmatrix}
	\]

	Hence, we conclude our proof. \\

	\section*{}

	Since, for $j\in\mathbb{Z}^{+}$,

	\[	
		A^j
		=
		\begin{bmatrix}
			\cos\frac{2j\pi}{n} & -\sin\frac{2j\pi}{n} \\
			\\
			\sin\frac{2j\pi}{n} &  \cos\frac{2j\pi}{n}
		\end{bmatrix}
	\]

	It is obvious that when $j = n$, $A^j$ decomposes to:

	\[
		\begin{bmatrix}
			\cos 2\pi & -\sin 2\pi \\
			\sin 2\pi &  \cos 2\pi
		\end{bmatrix}
		=
		\begin{bmatrix}
			1 & 0 \\
			0 & 1 \\
		\end{bmatrix}
		= I = A^0
	\]

	So, $A^j = I$ when $|j| = n$. \\


	\subsection*{Lemma: $A^j = A^{j+n}$}

	\[
		A^{j+n} = A^jA^n = A^jI = A^j
	\]

	\section*{}

	We now know that $A^j = A^{j+n}$. That is to say, $A^j$ repeats for every $n$ increase in its exponent. \\
	Hence, the group generated by $A^j$ has unique elements for $j = (1+kn, 2+kn, 3+kn, .., n+kn)$ where $k\in\mathbb{Z}$. Hence, it has $n$ unique elements.  
	\\

	\section*{Proving the group generated by B has 2 elements}

	\[
		B = 
		\begin{bmatrix}
			1 & 0 \\
			0 & -1
		\end{bmatrix}
	\]

	\[
		B^2 = 
		\begin{bmatrix}
			1 & 0 \\
			0 & -1
		\end{bmatrix}
		\begin{bmatrix}
			1 & 0 \\
			0 & -1
		\end{bmatrix}
		= 
		\begin{bmatrix}
			1 & 0 \\
			0 & 1
		\end{bmatrix}
		= I
	\]

	We can now write now write $B^3$ as $BB^2 = BI = B$.
	\\ This pattern continues, and this $B^k = B$ for all $k$ that are odd, and $B^k = I$ for all $k$ that are even. 
	\\
	Hence, the group generated by $B$ has 2 elements.

	\section*{Proving $A^jB = BA^{-j}$}

	We can see that this relationship is true for $j=1$

	First, we note that: 
	\[
		A^{-1} = 
		\begin{bmatrix}
			\cos\frac{2\pi}{n} & \sin\frac{2\pi}{n} \\
			\\
			-\sin\frac{2\pi}{n} & \cos\frac{2\pi}{n}
		\end{bmatrix}
	\]

	Then, 

	\[
		BA^{-1} = 
		\begin{bmatrix}
			1 & 0 \\
			0 & -1
		\end{bmatrix}
		\begin{bmatrix}
			\cos\frac{2\pi}{n} & \sin\frac{2\pi}{n} \\
			\\
			-\sin\frac{2\pi}{n} & \cos\frac{2\pi}{n}
		\end{bmatrix}
		= 
		\begin{bmatrix}
			\cos\frac{2\pi}{n} & \sin\frac{2\pi}{n} \\
			\sin\frac{2\pi}{n} & -\cos\frac{2\pi}{n}
		\end{bmatrix}
	\]

	\[
		= 
		\begin{bmatrix}
			\cos\frac{2\pi}{n} & -\sin\frac{2\pi}{n} \\
			\sin\frac{2\pi}{n} &  \cos\frac{2\pi}{n}
			\end{bmatrix}
			\begin{bmatrix}
				1 & 0 \\
				0 & -1
			\end{bmatrix}
			=
			AB
		\]

Then, we can see that 
\[
	BA^{-2} = BA^{-1}A^{-1} = ABA^{-1} = AAB = A^2B
\]

Assume this is true for $j = 1,2,3,4,...,x-1$ \\
\\
Then, for $j=x$, we see that:

\[
	A^xB = AA^{x-1}B = ABA^{1-x} = BA^{-1}A^{1-x} = BA^{-x}
\]
Hence, we have proven the relationship by PCI.

\section*{Conclusion}

Hence, we can show that the group generated by A and B has 2n elements. The group has all unique $A^jB^k$ and $B^kA^j$. \\
However, we have proven that $B^kA^j = A^{-j}B^k$, so we can instead consider all unique $A^jB^k$ and $A^{-j}B^k$. \\
We have also proven that any $A^j$ can be written as $A^{j+n}$. So, any $A^j$ with a negative value for $j$ can be written instead with a positive exponent. \\
So, we are only concerned with $A^jB^k$. We know that $A^j$ has n possible values, as we saw in our earlier proof(it is unique for $j = 1,2,3,4,...,n$). We also have proven that $B^k$ has two possible values. Therefore, the group generated by A and B have $2n$ possible values.



	\end{document}
